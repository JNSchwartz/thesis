\documentclass[
	%parspace, % Add vertical space between paragraphs
	%noindent, % No indentation of first lines in each paragraph
	%nohyp, % No hyphenation of words
	%twoside, % Double sided format
	%draft, % Quicker draft compilation without rendering images
	%final, % Set final to hide todos
]{elteikthesis}[2024/04/10]

% The minted package is also supported for source highlighting
% See elteikthesis_minted.tex for example
%\usepackage[newfloat]{minted}

% Document's metadata
\title{Auto-Scaling with Infrastructure-as-Code for Kubernetes-Driven Architectures} % title
\date{2024} % year of defense

% Author's metadata
\author{József Novák-Schwartz}
\degree{Mathematics BSc}

% Superivsor(s)' metadata
\supervisor{Győző Horváth} % internal supervisor's name
\affiliation{Associate Professor} % internal supervisor's affiliation
\extsupervisor{Zoltán Zvara} % external supervisor's name
\extaffiliation{Senior Developer} % external supervisor's affiliation

% University's metadata
\university{Eötvös Loránd University} % university's name
\faculty{Faculty of Informatics} % faculty's name
\department{Dept. of Software Technology and Methodology} % department's name
\city{Budapest} % city
\logo{elte_cimer_szines} % logo

% Add bibliography file
\addbibresource{elteikthesis.bib}

% The document
\begin{document}

% Set document language
%\documentlang{hungarian}
\documentlang{english}

% List of todos (not in the final document)
%\listoftodos[\todolabel]

% Title page (mandatory)
\maketitle
% Topic declaration page (mandatory) - can also be attached instead
%\includepdf{topicdeclaration.pdf}

% Table of contents (mandatory)
\tableofcontents
\cleardoublepage

% Main content
\chapter{Introduction}
\label{ch:intro}

What's the purpose of software development? 
As wikipedia states: Software development is the process used to create software. I would say this only used to be the process to create software. Today you don't have to develop anymore to create software. It's even kind of an antipattern to write code nowadays. We would like  to use reusable components with well defined behavior which are interchangeable and replaceble. Software development where you code is like to using a cypher or a poem to express something. We would like to avoid this kind of artisan component as much as possible. Definietly some components has to be chiveled the old way. But that starts to be a rare and occasinally used exception. The new way of software creation must be as close to human language as possible and closer to the people who bring value. Software created value starts to shrink with all the redundancy, numbness of software using customers, less low hanging fruits and the liberation of software creation. The no code movement kind of reached it's final stage with AI where human language engineering is nearly capable to exclude the computer engineers from the equasion. You can plan, design, create and host any idea without much engineering overhead which not only reduces cost on engineering hours but increases maintainabilty at the same time.
The main topic of my thesis is infrastructure as a code. The tool which elevates software from mathematics to practice is the hardware which hosts and enables software to accelerate. Hardware management traditionally developed like plant machine maintanence. Every software offering company become a small or huge plant owner who had to maintain its hardware plant, it's hardware maintaing employees and creating processes around the hardware anblement. This paradigm evolved naturally with the digital revolution.
Automation was key not only by the manufacturing side but it circled in the 
Software develoopment has to be rethought since the new paradigmes. Shifting round and around.


\cleardoublepage

\input{user.tex}
\cleardoublepage

\input{impl.tex}
\cleardoublepage

\input{sum.tex}
\cleardoublepage

% Acknowledgements (optional) - in case your thesis received funding or would like to express special thanks to someone
\chapter*{\acklabel}
\addcontentsline{toc}{chapter}{\acklabel}
In case your thesis received financial support from a project or the university, it is usually required to indicate the proper attribution in the thesis itself. Special thanks can also be expressed towards teachers, fellow students and colleagues who helped you in the process of creating your thesis.

% Appendices (optional) - useful for detailed information in long tables, many and/or large figures, etc.
\appendix
\input{sim.tex}
\cleardoublepage

% Bibliography (mandatory)
\phantomsection
\addcontentsline{toc}{chapter}{\biblabel}
\printbibliography[title=\biblabel]
\cleardoublepage

% List of figures (optional) - useful over 3-5 figures
\phantomsection
\addcontentsline{toc}{chapter}{\lstfigurelabel}
\listoffigures
\cleardoublepage

% List of tables (optional) - useful over 3-5 tables
\phantomsection
\addcontentsline{toc}{chapter}{\lsttablelabel}
\listoftables
\cleardoublepage

% List of algorithms (optional) - useful over 3-5 algorithms
\phantomsection
\addcontentsline{toc}{chapter}{\lstalgorithmlabel}
\listofalgorithms
\cleardoublepage

% List of codes (optional) - useful over 3-5 code samples
\phantomsection
\addcontentsline{toc}{chapter}{\lstcodelabel}
\lstlistoflistings
\cleardoublepage

% List of symbols (optional)
%\printnomenclature

\end{document}
